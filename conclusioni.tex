\chapter{Conclusioni e prospettive future}
\label{conclusioni}
A seguito dell'analisi qualitativa effettuata (sez.\ref{analisiQualitativa}), l'azienda presso la quale è stata svolta questa tesi
 ha ritenuto il lavoro fatto più che soddisfacente riconoscendo di aver acquisito tramite esso un sistema in grado di stimare l'assetto di un operatore con un margine di errore non significativo ai fini dell'applicazione e di aver contribuito ad arricchire il know-how aziendale riguardante la stima dell'assetto.\\
A partire dal lavoro di questa tesi vi sono molte prospettive, alcune delle quali verranno affrontate da proposte di tesi future:
\begin{itemize}
	\item Ottimizzazione dell'operazione di lettura delle misure dall'unità di misura inerziale tramite I2C.
	\item Caratterizzazione degli zero-offset e loro compensazione nell'algoritmo di \textit{Sensor Fusion} implementato.
	\item Identificazione della soluzione ottima per la realizzazione del modulo App.
\end{itemize}

Una volta realizzato interamente il sistema ideato e descritto nel capitolo \ref{descrizioeDelLavoro}, si avrà uno strumento innovativo in grado di fornire un aiuto significativo agli operatori che si trovano ad affrontare situazioni d'emergenza di questo tipo; contribuendo a salvare più vite umane.