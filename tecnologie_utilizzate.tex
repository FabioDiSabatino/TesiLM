\chapter{Unità di misura inerziale}
\label{tecnologie}
Le unità di misura inerziale \cite{mems} (in inglese Inertial Measurement Units - IMU) sono dispositivi elettronici basati su sensori inerziali come accelerometri (\ref{accell}) e giroscopi (\ref{giroscopi}). In molti casi a questi vengono aggiunti altri sensori utili ad applicazioni di navigazione come il magnetometro (\ref{magnetometro}). Nello specifico di questa tesi, l'IMU utilizzata è un circuito integrato composto da questi tre sensori (più altri non utilizzati come sensore di temperatura) realizzati tramite tecnologia MEMS (acronimo di Microelectro Mechanical System, ovvero sistemi meccanici microelettrici).\\
Nel corso degli anni l'interesse per questa tecnologia è cresciuto grazie ai vantaggi in termini economici e tecnici, tra questi i più importanti sono:
\begin{itemize}
	\item costo di realizzazione costante e proporzionale alla superficie del dispositivo
	\item grande potenziale di integrazione nei circuiti elettronici integrati
	\item basso consumo energetico
	\item dimensioni ridotte
\end{itemize}
 
 In questo capitolo si illustrano i principi di funzionamento alla base dei sensori integrati nell'IMU utilizzata nel lavoro di questa tesi e realizzati mediante tecnologia MEMS.


\section{Accellerometro}
\label{accell}

\section{Giroscopio}
\label{giroscopi}


\section{Magnetometro}
\label{magnetometro}