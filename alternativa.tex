\chapter{Realizzazione di uno Scenario}

Ora vedremo come i principi enunciati nel Capitolo \ref{cap:costruzione_scenario} vengono applicati facendo ricorso a degli scenari da me realizzati. \\
Si analizzeranno anche le istruzioni da impartire per avviare tutte le componenti necessarie ad una corretta esecuzione. Infine verranno mostrati alcuni Screenshot ed i risultati ottenuti dalle due emulazioni. Di seguito si presentano i due scenari.

\section{Gli scenari}

Si sono realizzati due scenari al fine di mostrare come tutti i punti descritti nel Capitolo \ref{cap:costruzione_scenario} portino effettivamente alla realizzazione di emulazioni perfettamente funzionanti. \\
Tutti i file di configurazione creati per dar vita a questi due esempi sono stati inseriti in Appendice \ref{app:a}. \\
Gli scenari proposti comprendono $8$ NEMs ($7$ in fila indiana fermi nella loro posizione iniziale e un ottavo nodo che cammina di fianco ad essi).QUANDO SI AVRANNO QUI SI POSSONO INSERIRE LE DISTANZE PRECISE TRA OGNI NODO E LA VELOCITA DELLA CAMMINATA. I $7$ nodi in fila vedono solamente i loro vicini stretti ($1$ vede solo $2$, $2$ vede $1$ e $3$, ..., $7$ vede solo $6$). L'ottavo cammina parallelamente a loro connettendosi di volta in volta solo e solamente con uno di essi (quello perpendicolare) fino ad arrivare a fine corsa staccandosi da tutti quanti. Ogni volta che si stabilisce una connessione (per esempio il nodo $8$ si connette con il nodo $2$ lasciando il numero $1$) la camminata dell'ottavo nodo si ferma per un tempo di $120$ secondi (ricordiamo che si svolge tutto in tempo reale). Quest'intervallo si è introdotto al fine di effettuare le misure necessarie e le dovute verifiche. \\
Di seguito si propone una rappresentazione grafica di quanto detto.

\begin{figure}[H]
	\centering
	\includegraphics[scale=0.6]{descrizione_scenario.png}
	\caption{Scenario delle Emulazioni Realizzate}
	\label{fig:descrizione_scenario}
\end{figure}

La differenza fondamentale che c'è tra i due esempi qui riportati risiede nel modello di propagazione (\textit{propagationmodel} \cite{propagationmodel}) utilizzato. Il primo scenario utilizza il modello di propagazione \textbf{freespace} mentre il secondo il \textbf{2-ray}. L'implementazione di EMANE di tali modelli segue un algoritmo per il calcolo dei \textit{pathloss} abbastanza ``semplice''. \\
Il \textit{freespace} segue la seguente espressione: 
\begin{equation*}
Pathloss[i++] = 20 log10[FSPL\cdot(\frac{FrequencyHz}{1000000})\cdot(\frac{Distance}{1000})]
\end{equation*} 
\noindent
Dove \textit{FSPL} = $41.916900439033640$ è una costante definita all'interno della libreria del modello di propagazione e \textit{Distance} è la distanza in metri tra i due nodi di cui si calcola il Pathloss. \\
A questo punto se il valore scaturito è maggiore di $0$ verrà attribuito un Pathloss pari al risultato altrimenti sarà nullo. La configurazione dello strato fisico secondo questo modello è quella in Appendice \ref{app:freespace}. \\

Il \textit{2-ray} invece segue un'altra espressione:
\begin{equation*}
Pathloss=20 log10(\frac{Distance^2}{this.node.altitude \cdot remote.node.altitude})
\end{equation*} 
\noindent
Anche qui se il risultato è minore di $0$ si pone un valore nullo altrimenti si pone il \textit{pathloss} uguale al valore risultante. \\
Questo modello di propagazione tiene conto anche dell'altitudine dei due nodi in quanto il secondo raggio viene riflesso a terra.

\begin{figure}[H]
	\centering
	\includegraphics[scale=5]{2ray.png}
	\caption{Modello di propagazione 2ray}
	\label{fig:2ray}
\end{figure}
\noindent
Lo strato fisico derivante dal modello di propagazione \textit{2-ray} è consultabile in Appendice \ref{app:2-ray}. \\

\section{Avvio dell'Emulazione}

Arrivati a questo punto abbiamo tutto pronto per far partire i nostri due scenari esempio. Si procede quindi ad una spiegazione di come avviare tutti gli $8$ containers e come lanciare all'interno di essi EMANE e tutti i suoi tools ausiliari. \\
Sono stati creati degli \textit{script} che rendono automatico l'avvio dell'emulazione (Appendice \ref{app:script_avvio}) e anche per il loro arresto (Appendice \ref{app:script_arresto}).

\subsection{Fase Preliminare}

Prima di poter iniziare effettivamente si ha bisogno di installare e soddisfare tutte le dipendenze in modo da poter eseguire tutti i tools in tranquillità. A tal fine si consiglia di seguire la piccola guida presente su questo link \url{https://github.com/adjacentlink/emane-tutorial/wiki/Install}.

\subsection{Avviare gli LXC Containers}

Per gli scenari esempio si è scelto di utilizzare dei containers ``applicazione'' per poter sfruttare il loro veloce avvio e soprattutto perché non c'è bisogno di dover creare anticipatamente il container. Con questo approccio una volta terminata l'emulazione, quindi arrestati i vari container, non rimane traccia sul \textit{File System} della loro esistenza. Il comando base da eseguire per lanciare un container è:
\begin{verbatim}
lxc-execute {-n name} [-f config-file] [-- command]
\end{verbatim}
\noindent
Dove \textit{name}, \textit{config-file} e \textit{command} sono rispettivamente il nome che si vuole dare al container, il file di configurazione XML (Appendice \ref{app:conf_lxc_nodo1}) e i comandi da eseguire all'interno del container. \\
Se non si sta utilizzando lo script automatico si consiglia di inserire nel campo \textit{command} l'avvio del demone ssh (\textit{sshd}) in modo da avere accesso tramite terminale al container.

\subsection{Comandi da eseguire internamente ai Container}

Una volta approntati tutti i container del nostro scenario si deve procedere all'avvio di tutti i servizi di cui abbiamo bisogno. Le operazioni di seguito indicate dovranno essere eseguite su ogni container dello scenario. \\

Come prima cosa, ovviamente, si lancia EMANE dandogli in input il file di configurazione NEM Platform Server (Appendice \ref{app:platform_server}) e quindi si avrà:
\begin{verbatim}
emane config-file -r -d -l 4 -f log-file
\end{verbatim}
\noindent
Dove \textit{-r} sta ad indicare priorità \textit{realtime} per noi essenziale in quando vogliamo che tutto sia emulato in tempo reale, \textit{-d} verrà eseguito in background e \textit{-l} indica l'accuratezza dei file di log. \\
A questo punto c'è già connettività tra i vari nodi della rete e quindi potremmo anche provare ad effettuare dei \textit{ping} diagnostici. Si può inoltre notare che è salita l'interfaccia \textit{emane0}.

\begin{figure}[H]
	\centering
	\includegraphics[scale=0.5]{ifconfig.png}
	\caption{Risultato ifconfig sul nodo-1 (troncato)}
	\label{fig:ifconfig}
\end{figure} 

A questo punto si lancia il demone degli eventi su ogni nodo della rete attraverso il comando:
\begin{verbatim}
emaneeventd config-file -r -d -l 4 -f log-file
\end{verbatim}
\noindent
Tutti i parametri da inserire hanno lo stesso significato di quelli precedenti. L'unica differenza sta nel fatto che, ovviamente, il file di configurazione è relativo all'Event Daemon (Appendice \ref{app:event_daemon}. \\

Un'altra componente molto importante per la riuscita dello scenario è sicuramente il \textit{ricevitore gps}. Il comando da impartire affinchè il demone \textit{gpsd} si avvii è il seguente:

\begin{verbatim}
gpsd -P pid-file -G -n -b file-read-only
\end{verbatim}
\noindent
INSERIRE QUI IL SIGNIFICATO DI QUESTI PARAMETRI.

Infine si procede all'avvio del protocollo di routing OLSR sempre tramite il suo demone (\textit{olsrd}). Il comando da impartire per eseguire ciò è:
\begin{verbatim}
olsrd -f config-file -d 3 > log-file
\end{verbatim}
\noindent
Dove \textit{config-file} è il file di configurazione presentato in Appendice \ref{app:routing}, \textit{-d $3$} indica il livello di \textit{debug} voluto. L'ultima parte del comando sta ad indicare il ridirezionamento sul file di log dell'output prodotto dal demone. \\

\subsection{Fase Finale}

Una volta eseguito quanto sopra tutti i nostri container sono pronti a ricevere gli eventi caratterizzanti il nostro scenario. Tornando sulla macchina host si deve eseguire il tool \textit{emaneeventservice} al fine di pubblicare gli eventi già dichiarati nel nostro file \textit{.eel} (Appendice \ref{app:eel_file}). Quindi procediamo impartendo quanto segue alla macchina host:
\begin{verbatim}
emaneeventservice config-file -l 3 > log-file
\end{verbatim}
\noindent
Dove \textit{config-file} è il file di configurazione già presentato in Appendice \ref{app:emaneeventservice}. Come prima \textit{-l 3} indica il livello di log desiderato da girare sul \textit{log-file}. \\
Una volta dato il seguente comando lo scenario vero e proprio ha inizio e si inizieranno a vedere le posizioni GPS da noi impostate e tutto il resto degli eventi programmati nel file \textit{.eel}.

\begin{figure}[H]
	\centering
	\includegraphics[scale=0.5]{scenario_funzionante.png}
	\caption{Screenshot Emulazione}
	\label{fig:scenario_funzionante}
\end{figure} 
