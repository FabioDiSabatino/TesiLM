\chapter{Descrizione del lavoro}
\label{scenario}
Per meglio comprendere la soluzione proposta e data la complessità del sistema ideato, utilizzeremo un approccio top-down per descrivere il lavoro svolto e il sistema realizzato. Partendo da un livello d'astrazione a "black box"(Fig.\ref{fig:livelliAstrazione}) in cui emergono solo i requisiti funzionali, scenderemo ad un livello nel quale si ha una visione dei sottosistemi e i requisiti vengono raffinati. Infine nell'ultimo livello avremo una visione dei moduli che compongono il sistema e che sono stati realizzati nel contesto di questa tesi.
\begin{figure}[H]
	\centering
	\includegraphics[scale=0.4]{DescrizioneDelSistema/livelli_astrazione.png}
	\caption{Rappresentazione dei livelli d'astrazione utilizzati per descrivere il sistema }
	\label{fig:livelliAstrazione}
\end{figure}
\newpage

\section{Livello black box}
Considerando il sistema in questione come una black box (Fig.\ref{fig:requisitiFunzionali}) e l'infrastruttura di rete in modo astratto, i due macro-requisiti funzionali  sono rispettivamente:
\begin{itemize}
	\item \textbf{R1}: Geolocalizzare l'operatore
	\item \textbf{R2}: Trasmettere messaggi predefiniti come stato della vittima e codici d'emergenza.
\end{itemize}
\begin{figure}[H]
	\centering
	\includegraphics[scale=0.3]{DescrizioneDelSistema/requisitiSistema.png}
	\caption{Rappresentazione del sistema come black box }
	\label{fig:requisitiFunzionali}
\end{figure}
Nelle fasi primordiali del progetto, si è scelto di allocare la maggior parte delle risorse lavorative nel completamento di \textbf{R1} lasciando ad una fase successiva lo sviluppo di \textbf{R2}, per questo motivo d'ora in avanti considereremo come unico requisito la geolocalizzazione dell'operatore. \textit{R1} può essere suddiviso in due requisiti più specifici:
\begin{itemize}
	\item \textbf{R1.1}: Determinare la posizione di un operatore all'interno della rete
	\item \textbf{R1.2}: Identificare il cammino minimo da un nodo ad un altro
\end{itemize}
Quest'ultimo rappresenta sia la possibilità da parte dell'operatore di eseguire il percorso all'inverso, sia la possibilità che venga raggiunto da una squadra di supporto. A questo punto possiamo scendere al livello d'astrazione successivo (\ref{fig:livelliAstrazione}).
\newpage 


\section{Livello sottosistemi}
Come già accennato nell'introduzione, la rete verrà costruita dinamicamente da un'esploratore e man mano che egli avanza verrà ampliata aggiungendo nuovi nodi.
Realizzare un'architettura di rete del genere introduce numerose problematiche, alcune strettamente legate alla tecnologia utilizzata (dettagliata nel capitolo successivo) altre alle caratteristiche richieste.\\
Nello specifico la problematica sulla quale il lavoro di questi tesi si è concentrato è la progettazione e lo sviluppo dei sottosistemi destinati alla georeferenziazione dei nodi, ovvero l'attribuzione dell'informazione riguardante la dislocazione geografica dei nodi rispetto alla rete.\\
In questo contesto i sottosistemi individuati nella fase di progettazione sono rappresentati dalla seguente figura: 
\begin{figure}[H]
	\centering
	\includegraphics[scale=0.3]{DescrizioneDelSistema/sistema_liv1.png}
	\caption{Rappresentazione del sistema in sottosistemi }
	\label{fig:sistema_liv1}
\end{figure}

Ognuno dei quali con la propria responsabilità:
\begin{itemize}
	\item \textbf{S1}: ha il compito di ricavare tramite la rete e altri sensori i dati necessari
	\item \textbf{App}: ha il compito di ricevere i dati da S1, elaborarli e infine fornire la posizione dell'operatore all'interno della rete o il percorso verso uno specifico nodo
\end{itemize}

La soluzione al problema si costruisce per iterazione georeferenziando i nodi nel momento in cui vengono aggiunti dall'operatore. Con riferimento all'esempio illustrato precedentemente (vedi Fig.\ref{fig:step1}- Fig.\ref{fig:step6}) consideriamo lo step 2.\\
Per ipotesi supponiamo che il primo nodo sia già georeferenziato, nel momento in cui risulti essere al limite della line-of-sight e/o della distanza di sicurezza (20 mt) l'operatore piazzerà il secondo nodo. Per georeferenziarlo (rispetto al primo) abbiamo bisogno di due informazioni fondamentali:
\begin{itemize}
	\item La distanza tra i due nodi
	\item L'angolo tra i due nodi
\end{itemize} 
Per mantenere il livello d'astrazione attuale ci basta sapere che le caratteristiche della tecnologia utilizzata nell'implementazione della rete, fa si che le singole celle abbiano un raggio d'azione all'interno del quale il sottosistema \textit{S1} può calcolare la distanza tra l'operatore e il centro della cella di appartenenza.\\
Tale informazione non è però sufficiente, infatti ci sono infiniti punti sulla circonferenza con centro nel primo nodo e raggio pari alla distanza. Per poter georeferenziare in modo univoco il secondo nodo abbiamo bisogno di trovare anche l'angolo in riferimento al primo nodo, tale compito non è banale e tanto meno le metodologie univoche e perfette.\\
La figura seguente rappresenta un tentativo di georeferenziare il secondo nodo utilizzando soltanto la distanza tra i due nodi.
\begin{figure}[H]
	\centering
	\includegraphics[scale=0.4]{DescrizioneDelSistema/ambiguitDistanza.png}
	\caption{Ambiguità nella georeferenziazione di un secondo nodo utilizzando soltanto la distanza dal precedente }
	\label{fig:ambiguitDistanz}
\end{figure}
Nell'esempio appena proposto, abbiamo mostrato solo tre delle possibili infinite posizioni del secondo nodo. Assumiamo che il punto corretto sia l'intersezione tra il vettore in blu e la circonferenza. In tal caso un'ambiguità con il vettore in verde potrebbe essere accettata in quanto si discosta di pochi metri dalla reale posizione, ben diversa sarebbe un'ambiguità con il vettore in giallo che renderebbe l'informazione del tutto errata e il sistema disinformante.\\
Per il momento ci basta sapere che l'algoritmo utilizzato nel contesto di questa tesi e approfondito nei capitoli successivi (vedi rif. algoritmo di fusione), per evitare questa ambiguità nel migliore dei modi, esegue un campionamento durante il tragitto tra un nodo e l'altro combinando i dati inerenti alla distanza dell'operatore con quelli relativi agli angoli lungo il percorso.\newpage
Nel prossimo paragrafo ci porteremo al livello d'astrazione più basso della nostra piramide (vedi \ref{fig:livelliAstrazione}) dettagliando i moduli che compongono i due sottosistemi e illustrando come questi intendono risolvere il problema di georeferenziazione appena esposto.


\section{Livello moduli}

A questo livello d'astrazion
\begin{figure}[H]  
\includegraphics[scale=0.3 ]{DescrizioneDelSistema/sistema_liv2.png}
\caption{Rappresentazione del sistema come composizione di moduli}
\label{fig:PropProf}
\end{figure}

