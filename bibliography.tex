\begin{thebibliography}{99}
\addcontentsline{toc}{chapter}{Bibliografia}

\bibitem{NOTE}
Theoretical Notes on Vulnerability to Disaster
\url{http://emergency-planning.blogspot.it/2009/01/theoretical-notes-on-vulnerability-to.html}

\bibitem{COMMON}
Fifty-six Common Misconceptions About Disaster
\url{http://emergency-planning.blogspot.it/2008/12/forty-four-common-misconceptions-about.html}

\bibitem{FASI_MANAGEMENT}
Phases of Emergency Management
\url{http://www.cheltenhamtownship.org/mobile/pview.aspx?id=4161&catid=29}

\bibitem{RESPONSE}
Disaster Response: a Multi-Agent based Approach,
\url{http://www.sapienzaapps.it/chitaly2015/wp-content/uploads/2015/09/02_chitaly2015_Costantini_et_al.pdf}

\bibitem{LICENZA_OSM}
Open Data Commons  \\
\url{http://opendatacommons.org/licenses/odbl/summary/}

\bibitem{MAPPE_LIBERE}
wiki.openstreetmap, ``Perché state realizzando OSM?'' \\
\url{http://wiki.openstreetmap.org/wiki/IT:FAQ}

\bibitem{WHYNEED}
Why the world needs OpenStreetMap, ``Perché il mondo ha bisogno di OSM?''
\url{http://www.theguardian.com/technology/2014/jan/14/why-the-world-needs-openstreetmap}

\bibitem{ACCURATEZZA_MAPPE}
wiki.openstreetmap, ``Com'è possibile che un progetto del genere porti a mappe accurate?'' \\
\url{http://wiki.openstreetmap.org/wiki/IT:FAQ}

\bibitem{GOOGLE_PLAN}
Google Developers ,  ``JavaScript API Usage Limits''
\url{https://developers.google.com/maps/pricing-and-plans/}


\bibitem{GOOGLE_OFFLINE}
Google Support ,  ``Download e utilizzo di aree offline''
\url{https://support.google.com/gmm/answer/6291838?hl=it}

\bibitem{LINK_PLANET}
Wiki OSM,  ``Planet.OSM"
\url{https://wiki.openstreetmap.org/wiki/Planet.osm}

\bibitem{HOT_PROJECT}
Humanitarian Openstreetmap Team,  ``Disaster Mapping"
\url{https://hotosm.org/projects/disaster-mapping}

\bibitem{WIKI_HAITI}
Terremoto di Haiti del 2010
\url{https://it.wikipedia.org/wiki/Terremoto_di_Haiti_del_2010}

\bibitem{EXPERIENCE}
User Experience
\url{https://it.wikipedia.org/wiki/User_Experience}

\bibitem{LEAFLET}
Leaflet: an open-source JavaScript library for mobile-friendly interactive maps
\url{https://it.wikipedia.org/wiki/User_Experience}

\bibitem{GROTTE}
Storia della cartografia
\url{https://it.wikipedia.org/wiki/Storia_della_cartografia}

\bibitem{REST}
I principi dell’architettura RESTful
\url{http://www.html.it/pag/19596/i-principi-dellarchitettura-restful} 




\end{thebibliography}



