\chapter{Stima dell'assetto tramite sensor fusion (NON completo)}
\label{elaborazione}
In questo capitolo vengono inizialmente illustrati gli strumenti matematici utilizzati per rappresentare l'assetto di un corpo rigido nello spazio,
successivamente viene illustrato l'algoritmo di fusione dei dati, provenienti dall'unità di misura inerziale, utilizzato per la stima dell'assetto dell'operatore.


\section{Rappresentazione geometrica dell'assetto di un corpo rigido nello spazio}
\label{assetto}
Con \textit{"assetto di un corpo rigido"} si intende l'orientamento di un corpo rigido rispetto ad un particolare sistema di riferimento.\\
Tale orientamento è rappresentato da una matrice di rotazione che, applicata ad un qualsiasi vettore nel sistema di riferimento mobile, ne fornisce una rappresentazione nel sistema di riferimento fisso (\ref{modello_di_misura}). 
Tale rotazione può essere espressa attraverso numerosi strumenti matematici, tra i più utilizzati si hanno:
\begin{itemize}
	\item \textbf{Angoli di Eulero}
	\item \textbf{Quaternioni unitari}
\end{itemize}

Nella Tab.\ref{rappresentazioni} vengono riportate sinteticamente le caratteristiche delle rappresentazioni appena enunciate \cite{assetto}:

\begin{table}[H]
	\centering

	\label{rappresentazioni}
	\begin{tabular}{lllll}
		\cline{1-3}
		\multicolumn{1}{|l|}{\textbf{Rappresentazione}} & \multicolumn{1}{l|}{\textbf{\#Parametri}} & \multicolumn{1}{l|}{\textbf{Caratteristiche}}                                                                                                                                                                                                                                                    &  &  \\ \cline{1-3}
		\multicolumn{1}{|l|}{Angoli di Eulero}          & \multicolumn{1}{c|}{3}                    & \multicolumn{1}{l|}{\begin{tabular}[c]{@{}l@{}}- facilmente interpretabili \\ dall'essere umano\\ \\ - funzioni trigonometriche nelle \\ relazioni cinematiche\\ \\ -soffrono del fenomeno \\ noto come \textit{Gimbal lock}\\ \\ - meno accurati dei quaternioni\end{tabular}}                           &  &  \\ \cline{1-3}
		\multicolumn{1}{|l|}{Quaternioni unitari}       & \multicolumn{1}{c|}{4}                    & \multicolumn{1}{l|}{\begin{tabular}[c]{@{}l@{}}- non interpretabili facilmente\\ dall'essere umano\\ \\ - equazioni della cinematica \\ lineari\\ \\ - costo computazione di \\ elaborazione minore degli\\ angoli di Eulero\\ \\ - necessitano di un vincolo di \\ norma unitaria\end{tabular}} &  &  \\ \cline{1-3}
		&                                           &                                                                                                                                                                                                                                                                                                  &  & 
	\end{tabular}
	\caption{Tabella comparativa delle rappresentazioni d’assetto}
\end{table}
Nell'algoritmo di fusione dei dati, dettagliato nei paragrafi successivi, si è adottato un approccio ibrido molto comune nei contesti applicativi dei sistemi IPS(\ref{IPS}). Tale approccio consiste nell'utilizzare la rappresentazione mediante \textit{quaternioni} per le computazioni mentre la rappresentazione mediante gli \textit{angoli di Eulero} per la visualizzazione.\\

\subsection{Matrice di rotazione}
Di seguito si fornisce una definizione generale di matrice di assetto\cite{assetto2}. Si supponga di avere due sistemi di riferimento cartesiani in tre dimensioni $F$ e $G$, una matrice ortogonale $A_{FG}$, detta di rotazione ed un vettore $x_G$ espresso nel sistema di riferimento $G$.\\
La matrice $A_{FG}$ permette di esprimere il vettore $x_G$ rispetto al sistema di riferimento $F$ secondo la seguente equazione:

\begin{equation}
x_F = A_{FG}x_G
\end{equation}
Essendo per ipotesi la matrice $A_{FG}$ ortogonale, l'operazione di inversione corrisponde al calcolo della sua trasposta:
 \begin{equation}
 x_G = A_{FG}^Tx_F
 \end{equation}
 Quindi determinare l'assetto significa definire la matrice di rotazione che permette, attraverso una semplice moltiplicazione, di ruotare i vettori da un sistema di riferimento mobile ad uno fisso.\\
 

\subsection{Angoli di Eulero}
\label{angoliEulero}
Si consideri \cite{assetto2} un sistema di riferimento cartesiano $F$  fisso (con assi $x_F$,$y_F$ e $z_F$) ed un sistema di riferimento cartesiano $G$ mobile (con assi $x_G$,$y_G$ e $z_G$). \\
Affinché l'orientamento degli assi del sistema mobile $G$ coincida con quelli del sistema fisso $F$, si devono eseguire almeno tre rotazioni successive attorno ai tre assi.\\
Tale vincolo è posto dal teorema di Eulero che è alla base di tutte le matrici di rotazioni. Il teorema afferma che:
\begin{itemize}
	\item Per ogni rotazione, esiste sempre un vettore che avrà la medesima rappresentazione nei due sistemi di riferimento
	\item ogni rotazione avviene sempre attorno ad un asse fisso
\end{itemize} 
L'idea è quella di ruotare ogni volta il sistema attorno ad uno dei suoi tre assi, così facendo l'asse attorno al quale è avvenuta la rotazione rimane fisso, mentre gli altri due cambiano orientamento. La rotazione successiva verrà fatta attorno ad uno dei due precedenti assi che hanno mutato l'orientamento. Con sistemi cartesiani a tre assi è possibile quindi scegliere tra dodici differenti sequenze di rotazioni per un totale eguale di possibili rappresentazioni della matrice di rotazione.\\
Nell'ambito di questa tesi e più comunemente in quello aeronautico, si è utilizzata la sequenza di rotazioni $z$-$y$-$x$ e gli angoli $\psi$,$\vartheta$ e $\varphi$ chiamati rispettivamente \textit{imbardata,beccheggio e rollio} (in inglese \textit{yaw,pitch e roll}) mostrati in Fig.\ref{fig:rpy}:
\begin{figure}[H]  
	\centering 
	\includegraphics[scale=0.5]{elaborazione/rpy.png}
	\caption{Rappresentazione degli angoli di roll, pitch e yaw per un velivolo \cite{wikiRoll}}
	\label{fig:rpy}
\end{figure}
Le tre rotazioni in questione sono:

\begin{equation}
A(z,\psi)= \begin{bmatrix}
\cos\psi   &-\sin\psi & 0 \\
\sin\psi     & \cos\psi  & 0 \\
0      & 0 & 1
\end{bmatrix}
\end{equation}

\begin{equation}
A(y,\vartheta)= \begin{bmatrix}
\cos\vartheta   &0 & \sin\vartheta \\
0    & 1  & 0 \\
-\sin\vartheta     & 0 & \cos\vartheta
\end{bmatrix}
\end{equation}


\begin{equation}
A(x,\varphi)= \begin{bmatrix}
1   &0 & 0 \\
0    & \cos\varphi  & -\sin\varphi\\
0     & \sin\varphi & \cos\varphi
\end{bmatrix}
\end{equation}

Andando a moltiplicare le precedenti matrici si ottiene la matrice di rotazione cercata:

\begin{eqnarray}
\label{matriceRotazione}
A(\psi,\vartheta,\varphi)= \begin{bmatrix}
\cos\psi \cos\vartheta  & \cos\psi \sin\vartheta \sin\varphi-\sin\psi \cos\varphi & \cos\psi \sin\vartheta \cos\varphi + \sin\psi \sin\varphi \\
\sin\psi \cos\vartheta    & \sin\psi \sin\vartheta \sin\varphi+\cos\psi \cos\varphi & \sin\psi \sin\vartheta \sin\varphi - \cos\psi \sin\varphi\\
-\sin\vartheta    & \cos\vartheta\sin\varphi & \cos\vartheta\cos\varphi
\end{bmatrix}
\end{eqnarray}

Il significato geometrico si ha osservando la Fig.\ref{fig:eulero}, dove:
\begin{itemize}
	\item la linea dei nodi è definita come l'intersezione tra il piano individuato dagli assi $x_Fy_F$ e quello individuato dagli assi $y_Bz_B$
	\item $\psi$ è l'angolo tra $y_F$ e la linea dei nodi
	\item $\vartheta$ è l'angolo tra $x_B$ e la sua proiezione sul piano $x_Fy_F$
	\item $\varphi$ è l'angolo tra $y_B$ e la linea dei nodi
\end{itemize}

\begin{figure}[H]  
	\centering 
	\includegraphics[scale=0.8]{elaborazione/eulero.png}
	\caption{Significato geometrico degli angoli di Eulero \cite{assetto2}}
	\label{fig:eulero}
\end{figure}
Come accennato nella tabella \ref{rappresentazioni}, la rappresentazione dell'assetto mediante angoli di Eulero presenta un problema di singolarità noto come \textit{gimbal lock}.\\
Nel caso della matrice di rotazione ricavata nell'equazione \ref{matriceRotazione}, il problema si presenta per $\vartheta = \pm \dfrac{\pi}{2}$: in questa situazione esistono infinite combinazioni di $\varphi$ e $\psi$ che portano alla stessa matrice di rotazione, nel caso di $\vartheta= \dfrac{\pi}{2}$ si ha:
\begin{equation}
\label{matriceGimbal}
A(\psi,\vartheta,\varphi)= \begin{bmatrix}
0   &\sin(\varphi - \psi) & \cos(\varphi - \psi) \\
0    & \cos(\varphi-\psi)  &-\sin(\varphi - \psi)\\
-1    & 0 & 0
\end{bmatrix}
\end{equation}
Ottenuta mediante sostituzione e applicazione delle formule di addizione e sottrazione di seno e coseno. Mentre la matrice di rotazione in Eq.\ref{matriceRotazione} permette una rotazione completa attorno ad un qualsiasi asse, la matrice in \ref{matriceGimbal} permette la rotazione attorno al solo asse $X$. Quindi si manifesta un vincolo di rotazione con conseguente perdita di un grado di libertà, come mostrato in Fig.\ref{fig:gimbal}. \\
Basti pensare ad un velivolo con un \textit{pitch} di $90°$, modificare il \textit{roll} o lo \textit{yaw} di un certo angolo avrebbe il medesimo effetto.\\
\begin{figure}[H]  
	\centering 
	\includegraphics[scale=0.3]{elaborazione/gimbal.png}
	\caption{Rappresentazione del fenomeno di \textit{gimbal lock} \cite{gimbal}}
	\label{fig:gimbal}
\end{figure}



\subsection{Quaternioni unitari}
\label{quaternioni}
I quaternioni hanno la peculiarità di essere il metodo di rappresentazione dell'assetto con minor numero di parametri privi di singolarità, come ad esempio il \textit{gimbal lock} visto precedentemente per gli angoli di Eulero.\\
Il quaternione unitario è un vettore composto da tre elementi che definiscono il vettore \textbf{$q_{1:3}$} e da un elemento scalare $q_4$, tale che la norma:
\begin{equation}
 ||\overrightarrow{q}|| = \sqrt{q_1^2 + q_2^2 + q_3^2 + q_4^2}=1
\end{equation}
Questi possono essere usati per rappresentare l'assetto di un corpo rigido in quanto le trasformazioni che legano le terne di Eulero ai quaternioni unitari, sono semplicemente delle trasformazioni algebriche che portano da uno spazio rappresentativo all'altro.\\
Sfruttando il teorema di Eulero \cite{assetto2}, si può parametrizzare la matrice di rotazione in \ref{matriceRotazione} ottenendo l'equivalente in funzione dei quaternioni:
\begin{equation}
\label{matriceQuaternioni}
A(\overrightarrow{q})= \begin{bmatrix}
 q_1^2 -  q_2^2 -  q_3^2 -  q_4^2 & 2(q_1 q_2 + q_3  q_4) & 2(q_1 q_3 - q_2 q_4) \\
2(q_2 q_1 - q_3 q_4)    &  -q_1^2 + q_2^2 -  q_3^2 +  q_4^2  & 2(q_2 q_3 + q_1q_4)\\
2(q_3 q_1 + q_2 q_4)    & 2(q_3 q_2 - q_1 q_4)  &  -q_1^2 -  q_2^2 +  q_3^2 +  q_4^2
\end{bmatrix}
\end{equation}
Come si può notare dalla \ref{matriceRotazione}, i quaternioni permettono di definire una matrice di assetto i cui elementi sono funzioni quadratiche omogenee degli elementi del quaternione. Si evita così qualsiasi tipo di calcolo trigonometrico (e relative singolarità) e si ha un costo computazionale minore.\\

Un diretto confronto tra le matrici \ref{matriceRotazione} e \ref{matriceQuaternioni} permette inoltre di definire il rapporto tra gli angoli di Eulero ($\psi,\vartheta,\varphi$) e le componenti del quaternione ($q_1, q_2, q_3,q_4$) attraverso le seguenti:

\begin{equation}
\psi = Atan2 \left( \frac{2q_2q_3 - 2q_1q_4}{2q_1^2+q_2^2 -1}\right)
\end{equation}

\begin{equation}
\vartheta = -\sin(2q_2q_4 + 2q_1q_3)
\end{equation}

\begin{equation}
\psi = Atan2\left(  \frac{2q_3 q_4 - 2q_1q_2}{2q_1^2+q_4^2 -1}\right)
\end{equation}


\section{Algoritmo di sensor fusion per la stima dell'assetto}
\label{sensor_fusion}
La maggior parte degli algoritmi di stima dell'assetto che sfruttano i dati provenienti da sensori (sez.\ref{tecnologie}) si basano sull'applicazione di un \textit{filtro di Kalman}.\\
Il filtro di Kalman è un efficiente filtro ricorsivo che valuta lo stato di un sistema dinamico a partire da una serie di misure soggette a rumore. Per le sue caratteristiche intrinseche è un filtro ottimo per rumori e disturbi agenti su sistemi gaussiani a media nulla \cite{kalmanWiki}.\\
Con riferimento alla rappresentazione del sistema realizzato nell'ambito di questa tesi (sez.\ref{livello_moduli}), all'interno del "modulo" \textit{microcontroller} e del "modulo" \textit{App} vi sono due diversi algoritmi di \textit{sensor fusion} basati sul filtro di Kalman che, a partire dai dati grezzi letti dall'unità di misura inerziale (sez.\ref{tecnologie}), stimano l'assetto relativo al modello di misura (sez.\ref{modello_di_misura}).\\
Il modulo \textit{Microcontroller} utilizza una libreria esterna realizzata da \textit{STM} \cite{motion} della quale però non vengono forniti i dettagli implementativi, perciò nei paragrafi successivi si farà riferimento all'algoritmo utilizzato all'interno del modulo \textit{App} che è  l'implementazione di un lavoro di tesi trovato in letteratura \cite{trackingThesis}.

\subsection{Equazioni di stato del sistema}

 E' necessario come prima cosa, descrive l'evoluzione del sistema con un equazione differenziale, dove lo stato del sistema è rappresentato tramite quaternioni unitari (\ref{quaternioni}).\\
 Definiamo con $\omega_{nb}^n$ il vettore delle velocità angolari rispetto al \textit{b-frame} dell'IMU (sez.\ref{modello_di_misura}):
 
\begin{equation}
\label{vettoreAngular}
\omega_{nb}^b  = [\omega_x, \omega_y, \omega_z]^T
\end{equation}

Dall'algebra dei quaternioni è possibile scrivere l'equazione che descrive l'evoluzione del sistema come:

\begin{equation}
\label{stateEquation}
\dot{q}_{n}^b = \frac{1}{2} \Omega_{nb}^n q_n^b
\end{equation}
Dove $\Omega_{nb}^n$ è la matrice con gli elementi del vettore della velocità angolare in \ref{vettoreAngular}:
\begin{equation}
\Omega_{nb}^n= \begin{bmatrix}
0 &  -\omega_x &  -\omega_y & -\omega_z\\
\omega_x     & 0  & \omega_z &  -\omega_y  \\
 \omega_y & -\omega_z  & 0 &  \omega_x \\
 \omega_z & \omega_y & -\omega_x & 0
\end{bmatrix}
\end{equation}

Si noti bene come dalla precedente equazione \ref{stateEquation} l'evoluzione del sistema, corrispondente alla fase detta "aggiornamento a priori" del filtro di Kalman, è caratterizzata dai soli dati del giroscopio. Come si vedrà tra poco, i sensori di accelerometro (sez.\ref{accell}) e magnetometro (sez.\ref{magnetometro}) verranno utilizzati nella fase detta "aggiornamento a posteriori" per correggere la stima ottenuta con il solo giroscopio.

\subsection{Filtro di Kalman a due stadi}
Un filtro di Kalman completo potrebbe essere computazionalmente troppo oneroso da eseguire per un dispositivo come un microcontrollore. Per questo motivo in \cite{trackingThesis} si è pensato di adottare un approccio diverso dall'implementazione classica del filtro. Tale metodo è ancora basato sul filtro di Kalman, ma la fase di correzione della stima ottenuta nell'aggiornamento a priori, viene divisa in due parti, ognuna delle quali agisce su una specifica componente.\\
Il primo stadio utilizza i dati provenienti dall'accelerometro (sez.\ref{accell}) per correggere gli angoli stimati di \textit{roll} e \textit{pitch} (sez.\ref{angoliEulero}) mentre il secondo stadio utilizza i dati provenienti dal magnetometro (sez.\ref{magnetometro}) per correggere la stima dello \textit{yaw}. Ogni stadio lavora con un singolo vettore di dati grezzi che è usato appunto per correggere una specifica parte della stima dell'assetto, da questo ne consegue che ogni stadio può lavorare con una matrice di dimensione $ 4 x 3$ contro le $ 4 x 6$ di un filtro di Kalman completo. Da questo ne consegue un guadagno notevole in termini di prestazioni.\\
 Nella figura seguente una rappresentazione a blocchi del filtro a due stadi:
\begin{figure}[H]  
	\centering 
	\includegraphics[scale=0.3]{elaborazione/filtro.png}
	\caption{Rappresentazione del principio di funzionamento del filtro di Kalman a due stadi \cite{trackingThesis}}
	\label{fig:filtro}
\end{figure}

Un altro grande vantaggio consiste nella flessibilità dell'algoritmo, questo infatti può essere utilizzato anche se si dispone di un IMU a 6DOF (privo del magnetometro) semplicemente "disattivando" lo \textit{stadio 2} in Fig.\ref{fig:filtro}, così facendo l'uscita dallo \textit{stadio 1} diventa l'uscita del sistema e viene retroazionata allo \textit{stadio 0} andando a correggere la prossima lettura.
 
\subsection{Descrizione dell'algoritmo}
Lo stato del sistema è, come già detto, rappresentato tramite quaternioni (sez.\ref{quaternioni}). Lo \textit{stadio 0} anche detto \textit{stima a priori} utilizza solo i dati grezzi provenienti dal giroscopio.
Fatto ciò, si deve stimare il vettore gravitazionale $h_1$ per poter correggere il \textit{roll} e il \textit{pitch} nello \textit{stadio 1}. Tale vettore è così definito:
\begin{eqnarray}
h_1(q_k)= \hat{q}= R_n^b \begin{bmatrix}
0  \\
0 \\
|q|
\end{bmatrix}= |g| \begin{bmatrix}
2q_1q_3 - 2q_0q_2  \\
2q_0q_1 + 2q_2q_3 \\
q_0^2 - q_1^2 - q_2^2 + q_3^3
\end{bmatrix}
\end{eqnarray}
Il vettore gravitazionale $g$, espresso secondo l'\textit{i-frame} (sez.\ref{modello_di_misura}), viene riportato nel \textit{b-frame} utilizzando la seguente matrice di rotazione:
\begin{equation}
R_n^b= \begin{bmatrix}
q_1^2 +  q_2^2 -  q_3^2 -  q_4^2 & 2(q_2 q_3 + q_1  q_4) & 2(q_2 q_4 - q_1 q_3) \\
2(q_2 q_3 - q_1 q_4)    &  q_1^2 - q_2^2 +  q_3^2 -  q_4^2  & 2(q_3 q_4 + q_1q_2)\\
2(q_2 q_4 + q_1 q_3)    & 2(q_3 q_4 - q_1 q_2)  &  q_1^2 -  q_2^2 - q_3^2 +  q_4^2
\end{bmatrix}
\end{equation}
Da $h_1$ si ricava quindi la matrice $H_{k1}$ del guadagno di Kalman che verrà utilizzata successivamente dallo \textit{stadio 1}:
\begin{eqnarray}
H_{K1}= \frac{\delta h_{1[i]}}{\delta q_j}=  \begin{bmatrix}
-2q_3 & 2q_4 & -2q_1 & 2q_2  \\
2q_2 & 2q_1 & 2q_4 & 2q_3\\
2q_0 & -2q_1 & -2q_2 & 2q_3
\end{bmatrix}
\end{eqnarray}  





























