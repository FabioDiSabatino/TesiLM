\chapter{Stima dell'assetto tramite sensor fusion}
\label{elaborazione}
In questo capitolo vengono inizialmente illustrati gli strumenti matematici utilizzati per rappresentare l'assetto di un corpo rigido nello spazio,
successivamente viene illustrato l'algoritmo di fusione dei dati, provenienti dall'unità di misura inerziale, utilizzato per la stima dell'assetto dell'operatore.


\section{Rappresentazione geometrica dell'assetto di un corpo rigido nello spazio}
\label{assetto}
Con \textit{"assetto di un corpo rigido"} si intende l'orientamento di un corpo rigido rispetto ad un particolare sistema di riferimento. Tale condizione può essere rappresentata attraverso numerosi strumenti matematici, tra i più utilizzati si hanno:
\begin{itemize}
	\item \textbf{Angoli di Eulero}
	\item \textbf{Quaternioni unitari}
\end{itemize}

Nella Tab.\ref{rappresentazioni} vengono riportate sinteticamente le caratteristiche delle rappresentazioni appena enunciate \cite{assetto}:

\begin{table}[]
	\centering

	\label{rappresentazioni}
	\begin{tabular}{lllll}
		\cline{1-3}
		\multicolumn{1}{|l|}{\textbf{Rappresentazione}} & \multicolumn{1}{l|}{\textbf{\#Parametri}} & \multicolumn{1}{l|}{\textbf{Caratteristiche}}                                                                                                                                                                                                                                                    &  &  \\ \cline{1-3}
		\multicolumn{1}{|l|}{Angoli di Eulero}          & \multicolumn{1}{c|}{3}                    & \multicolumn{1}{l|}{\begin{tabular}[c]{@{}l@{}}- facilmente interpretabili \\ dall'essere umano\\ \\ - funzioni trigonometriche nelle \\ relazioni cinematiche\\ \\ -presentato singolarità in \\ alcune situazioni\\ \\ - meno accurati dei quaternioni\end{tabular}}                           &  &  \\ \cline{1-3}
		\multicolumn{1}{|l|}{Quaternioni unitari}       & \multicolumn{1}{c|}{4}                    & \multicolumn{1}{l|}{\begin{tabular}[c]{@{}l@{}}- non interpretabili facilmente\\ dall'essere umano\\ \\ - equazioni della cinematica \\ lineari\\ \\ - costo computazione di \\ elaborazione minore degli\\ angoli di Eulero\\ \\ - necessitano di un vincolo di \\ norma unitaria\end{tabular}} &  &  \\ \cline{1-3}
		&                                           &                                                                                                                                                                                                                                                                                                  &  & 
	\end{tabular}
	\caption{Tabella comparativa delle rappresentazioni d’assetto}
\end{table}



\subsection{Angoli di Eulero}

\subsection{Quaternioni }

\section{Sensor fusion mediante un filtro di Kalman}
\label{sensor_fusion}