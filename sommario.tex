\chapter*{Sommario}
\markboth{Sommario}{Sommario}
\addcontentsline{toc}{chapter}{Sommario}
\thispagestyle{empty}

Lo scopo di questa tesi è lo sviluppo di un sistema per la stima dell'assetto di un'operatore in ambienti indoor privi del segnale GPS.\\
Il sistema deve essere in grado di stimare gli angoli assunti dall'operatore nel tragitto che egli effettua per raggiungere la persona da soccorrere; queste informazioni verranno poi utilizzate da un sistema più grande, che esula dal lavoro di questa tesi, al fine di creare una rete di nodi all'interno della quale 
si è in grado di localizzare e guidare i soccorritori.\\
Per realizzare questo sistema si è utilizzato un microcontrollore della STM con processore ARM-Cortex M4, un'unità di misura inerziale della STM dotato di giroscopio accelerometro e magnetometro e l'applicativo Matlab.\\
Per stimare l'assetto si utilizzano due algoritmi di \textit{sensor fusion}; uno trovato in letteratura ed implementato sull'applicativo Matlab, un altro facente parte di una libreria appositamente sviluppata da STM per i processori di questa famiglia.
\\
\\

\noindent
\begin{Huge}
\textbf{Abstract}
\end{Huge}
\\
\\
\noindent
The aim of this thesis is the development of a system for estimating the attitude of an operator in indoor environments where the GPS signal is lacking.
The system must be able to estimate the angles assumed by the operator during the path done to reach the person who has to be rescued; this information will then be used by a larger system, which fall outside the work of this thesis, in order to create a network of nodes within which
it is possible to locate and drive rescuers for the whole time of their operations. \\
To realize this system, it was used a microcontroller by STM with ARM-Cortex M4 processor, an inertial measurement unit by STM equipped with an accelerometer gyroscope and a magnetometer and the Matlab application.
To estimate the attitude two algorithms of \ textit {sensor fusion} are used; one found in the literature and implemented on the Matlab application, another one  which belongs to a library specially developed by STM for the processors of this family.
